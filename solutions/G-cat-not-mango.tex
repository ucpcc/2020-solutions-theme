\section{G. 그건 망고가 아니라 고양이예요}

\begin{frame} % No title at first slide
    \sectiontitle{G}{그건 망고가 아니라 고양이예요}
    \sectionmeta{
        \texttt{dfs, binary\_search}\\
        출제진 의도 -- \textbf{\color{acplatinum}Medium}
    }
    \begin{itemize}
        \item 제출 139번, 정답 8팀 (정답률 5.75\%)
        \item 처음 푼 팀: \textbf{여기가월파2020인가요} (월파, 치러, 왔어요), 68분
        \item 출제자: \texttt{evenharder}
    \end{itemize}
\end{frame}

\begin{frame}{\textbf{G}. 그건 망고가 아니라 고양이예요}
    \begin{figure}[h!]
        \centering
        \includegraphics[width=0.35\linewidth]{../images/mango/mango_sleepy.png}
    \end{figure}
    
    문제의 모티브가 된 망고님입니다. 풀이랑 상관은 없지만 귀엽습니다.
\end{frame}

\begin{frame}{\textbf{G}. 그건 망고가 아니라 고양이예요}
    $S$를 $\$$ 단위로 $S = S_0\$ S_1 \$ \cdots \$ S_m$처럼 쪼개서 표현하면 $M_i$를 다음과 같이 쓸 수 있습니다.
    \begin{itemize}
        \item $M_1 = S_0 \underline{M_0} S_1 \underline{M_0} \cdots \underline{M_0} S_m$
        \item $M_2 = S_0 \underline{M_1} S_1 \underline{M_1} \cdots \underline{M_1} S_m$
        \item $ \vdots $
        \item $M_k = S_0 \underline{M_{k-1}} S_1 \underline{M_{k-1}} \cdots \underline{M_{k-1}} S_m$
    \end{itemize}
    즉 $M_i$를 $S$의 글자에 대응시킬 수 있습니다.
\end{frame}

\begin{frame}{\textbf{G}. 그건 망고가 아니라 고양이예요}
    $S$에 $\$$가 1개 있으면 어떻게 될까요? $S = S_0\$ S_1$이라 하면,
    \begin{itemize}
        \item $M_1 = S_0 \underline{M_0} S_1$
        \item $M_2 = S_0 S_0 \underline{M_0} S_1 S_1$
        \item $ \vdots $
        \item $M_k = S_0^k \underline{M_0} S_1^k$
    \end{itemize}
    가 되어, 반복문 등으로 문제를 해결할 수 있습니다. ($S_i^k$는 $S_i$가 $k$번 반복된 문자열입니다.)
\end{frame}

\begin{frame}{\textbf{G}. 그건 망고가 아니라 고양이예요}
     $\$$가 2개 이상 있으면 다음과 같은 DFS 기반 \complexity{Qk\vert S \vert} 풀이를 생각할 수 있습니다.
    \begin{itemize}
        \item DFS를 통해 $M_i$의 글자를 조사하면서, $\$$면 $M_{i-1}$의 길이를, 아니면 1을 뺍니다.
        \item 남은 칸 수보다 $M_{i-1}$의 길이가 크면 목표 문자열이 포함되어 있으므로 재귀합니다.
        \item 원하는 부분문자열을 전부 확인할 때까지 반복합니다.
    \end{itemize}
    
\end{frame}

\begin{frame}{\textbf{G}. 그건 망고가 아니라 고양이예요}
    시간 복잡도를 어떻게 줄일 수 있을까요?
    
    $M_{i+1}$이 $M_{i}$보다 2배 이상 길기 때문에, $M_{60}$ 정도만 되어도 길이는 $10^{18}$을 넘깁니다.
    
    그러므로 적당한 60 이하의 $k'$을 잡아, 길이가 $10^{18}$ 이상인 $M_{k'}$만 조사해도 충분합니다.
    \begin{itemize}
        \item $S$가 $S_1\$\cdots$꼴일 때, $M_k = S_1^{k-k'} M_{k'} \cdots$ 꼴입니다 ($ k' \leq k $).
        \item 조건에 의해, $ S_1^{k-k'} M_{k'}$ 이후는 탐색할 필요가 없습니다.
    \end{itemize}
    그러므로 $S_1^{k-k'}$는 따로 처리하고, $M_{k'}$에서 DFS를 적용하면 됩니다.
\end{frame}

\begin{frame}{\textbf{G}. 그건 망고가 아니라 고양이예요}
    \begin{itemize}
        \item DFS에서 각 $M_i$를 다 순회할 필요도 없이, 이분 탐색을 통해 원하는 시작 위치를 효율적으로 찾을 수 있습니다.
        \item 방문하는 칸의 개수는 세그먼트 트리와 비슷하게 출력하는 문자의 개수에 비례하므로, 시간복잡도 \complexity{Qk'\lg(\vert S \vert) + \sum (b_i - a_i)}의 풀이가 완성됩니다.
        \item $M_0$를 순회할 때 잘못 처리하면 시간복잡도에 $\vert M_0 \vert$가 추가로 곱해질 수 있습니다.
    \end{itemize}
\end{frame}